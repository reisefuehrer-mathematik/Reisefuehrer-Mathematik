\documentclass[../aussagenlogik.tex]{subfiles}

\begin{document}
\begin{exercise}{intro}
    Welche der folgenden Aussagen sind atomar? Alle Antworten sind zu Begründen. Wenn eine Aussage nicht atomar ist, gib ihre atomaren Unteraussagen an. \emph{Erinnerung: atomare Aussagen sind Aussagen, die nicht weiter in Unteraussagen aufgeteilt werden können.}
    \begin{enumerate}
        \item Der Trank ist grün und giftig.
        \item Der Trank ist grün.
        \item Der Trank ist nicht giftig.
        \item Wenn der Trank nicht giftig ist, dann trinke ich ihn.
    \end{enumerate}
\end{exercise}

\begin{exercise}{intro}
    Übersetze die folgenden Aussagen in ihre aussagenlogischen Formeln indem du atomare Aussagen durch Variablen ersetzt und anschließend die Operatoren durch ihre mathematischen Symbole darstellst. \emph{Aufgabenteil a) ist als Beispiel gegeben.}
    \begin{enumerate}
        \item Ich trinke den Trank und ich bekomme Superkräfte.\\
        \textit{$T \land S$ mit $T: \text{\enquote{Ich trinke den Trank}}$ und $S: \text{\enquote{Ich bekomme Superkräfte}}$}
        \item Wenn ich den Trank trinke, bekomme ich Superkräfte.
        \item Ich trinke den grünen Trank.
        \item Der Trank ist ein Heiltrank genau dann, wenn er rot ist und nicht sprudelt.
        \item Der Trank ist giftig, wenn er grün ist.
    \end{enumerate}
\end{exercise}

%\begin{exercise}{difficult}
%    Wie könnte man die folgenden Aussagen in aussagenlogische Formeln übersetzen?
%    \begin{enumerate}
%        \item Der blaue Zauberer wohnt im mittleren Haus.
%        \item Der gelbe Zauberer wohnt 
%    \end{enumerate}
%\end{exercise}
\end{document}