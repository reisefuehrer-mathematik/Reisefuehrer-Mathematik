\def\pathToMain{../../buch/}
\documentclass{uebungsblatt}
\usepackage[utf8]{inputenc}
\usepackage[T1]{fontenc}
\usepackage[ngerman]{babel}

\usepackage{mathdef}

\sheet{Vorbereitungsblatt 9.4}
\title{Äquivalenzumformungen}
\topic{\getchaptername{aussagenlogik}}
\chapternum{\getchapternum{aussagenlogik}}

\begin{document}
\maketitle
\begin{contents}
    Semantische Äquivalenz, Äquivalenzumformungen, Involution, Kontraposition, De Morgansche Gesetze, Auflösen von Implikation und Äquivalenz
\end{contents}

\video{Semantisch äquivalente Aussagen}{4}{Kapitel \ref{ext:sec:abbildungen_intuition} (ab Seite \pageref{ext:sec:abbildungen_intuition})}{https://www.google.de}

\begin{definition}
    Zwei Aussagen $A,B$ heißen \textbf{semantisch äquivalent}, wenn diese für jede Belegung ihrer atomaren Unteraussagen, den gleichen Wahrheitswert haben. Wir notieren dies durch $A \equiv B$.
\end{definition}

\begin{theorem}
    Sind $A$ und $B$ Aussagen, dann gilt:
    \begin{align*}
        \tag{Involution}
        \lnot \lnot A \equiv& A\\
        \tag{1. De Morgansches Gesetz}
        \lnot (A \land B) \equiv& (\lnot A) \lor (\lnot B)\\
        \tag{2. De Morgansches Gesetz}
        \lnot (A \lor B) \equiv& (\lnot A) \land (\lnot B)\\
        \tag{Kontraposition}
        A \implies B \equiv& (\lnot B) \implies (\lnot A)\\
        \tag{Auflösen der Implikation}
        A \implies B \equiv & \lnot (A \land (\lnot B))\\
        \tag{Auflösen der Äquivalenz}
        A \iff B \equiv & (A \implies B) \land (B \implies A)
    \end{align*}
\end{theorem}

\subsection*{Aufgabenteil}

-'lückentext' zu sem äquivalenz (straight forward)

-gegeben zwei aussagen, mit wahrheitstabellen sem äquivalenz zeigen, 

-eine reihe von äquivalenzumformungen zeigen, man soll den namen der regeln angeben,
die verwendet wurden

-denn eigenständi zwei aussagen mit äquivalenzumformungen sem äquiv zeigen

-wieder zwei aussagen (wellensittich beispiel) mit whwtabelle zeigen, dass nicht
sem äquivalent


\mandala{mandala/mandala01}

\end{document}
