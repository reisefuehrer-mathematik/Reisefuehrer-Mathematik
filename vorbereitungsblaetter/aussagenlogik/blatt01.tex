\def\pathToMain{../../buch/}
\documentclass{uebungsblatt}
\usepackage[utf8]{inputenc}
\usepackage[T1]{fontenc}
\usepackage[ngerman]{babel}
\usepackage{multicol}

\usepackage{mathdef}
\usepackage{multicol}

\sheet{Vorbereitungsblatt 9.1}
\title{Einführung: Aussagenlogik}
\topic{\getchaptername{aussagenlogik}}
\chapternum{\getchapternum{aussagenlogik}}

\begin{document}
\maketitle
\begin{contents}
    Aussagen, Wahrheitswerte
\end{contents}

\video{Was ist eine Aussage?}{4}{Kapitel \ref{ext:sec:abbildungen_intuition} (ab Seite \pageref{ext:sec:abbildungen_intuition})}{https://www.google.de}

\begin{definition}
    Eine Aussage ist ein Satz, der keine Frage ist, den man aber entweder mit 
    \enquote{Ja, das ist \emph{wahr}} bejahen oder mit \enquote{Nein, das ist \emph{falsch}} 
    verneinen kann.
\end{definition}



\begin{definition}
    Eine Aussage ist \emph{wahr}, falls die Antwortmöglichkeit \enquote{Ja, das ist \emph{wahr}}
    passend ist. Ist jedoch \enquote{Nein, das ist \emph{falsch}} die passende
    Antwortmöglichkeit, dann ist die Aussage \emph{falsch}.
\end{definition}

\subsection*{Aufgabenteil}

\begin{exercise}
    Eine Aussage ist \textchoice{ein Satz,eine Variable} von \textchoice{dem,der} wir sagen können, ob \textchoice{er,sie} \textchoice{sinnvoll,wahr oder falsch,angemessen}  
    ist. Auch wenn man auf die Frage \enquote{Hat Deutschland mehr Einwohner als Frankreich} mit \enquote{Ja, das ist wahr} antworten kann, ist sie \textchoice{eine,keine} Aussage. 
    
    Dafür ist aber der Satz \enquote{Deutschland hat mehr Einwohner als Frankreich} \textchoice{eine,keine} Aussage. Dieser Aussage können wir den Wahrheitswert \textchoice{\emph{wahr},\emph{falsch}} zuordnen, da \textchoice{Ja{,} das ist \emph{wahr}, Nein{,} das ist \emph{falsch}} die passende Antwort ist.
\end{exercise}

\begin{exercise}
    Welche der folgenden Formulierungen sind Aussagen?

    \begin{multicols}{2}
        \begin{multiplechoice}
            \item Regnet es heute?
            \citem Morgen fällt die Schule aus.
            \citem Kängurus sind Reptilien.
            \item Schau aus dem Fenster!
        \end{multiplechoice}
    \end{multicols}

\end{exercise}

\begin{exercise}

    Ordne diesen Aussagen ihren Wahrheitswert zu.

    \begin{multicols}{2}
        \begin{enumerate}[label=\alph*)]
            \citem Die Zahl 3 ist ungerade
                \begin{multiplechoice}
                        \item Wahr
                        \item Falsch
                \end{multiplechoice}
            \item Amerika gehörte zum römischen Reich
                \begin{multiplechoice}
                        \item Wahr
                        \item Falsch
                \end{multiplechoice}
            \item Füchse haben Kiemen 
                \begin{multiplechoice}
                        \item Wahr
                        \item Falsch
                \end{multiplechoice}
            \citem Göthe war ein Dichter
                \begin{multiplechoice}
                        \item Wahr
                        \item Falsch
                \end{multiplechoice}
        \end{enumerate}
    \end{multicols}

\end{exercise}    

\begin{exercise}
    Formuliere zwei \emph{wahre} und zwei \emph{falsche} Aussagen zu folgendem Text.
    \\ \\
    \emph{Tobias ist ein vorbildlicher Schüler. Tobias steht jeden Tag um 7:00 auf und putzt sich sofort die Zähne. Nachdem er noch am frühen Morgen seiner Mutter im Haushalt hilft, fährt er mit seinem Fahrrad in die Schule. Seine Lieblingsfächer sind Mathematik und Biologie, wobei er Deutsch eher nicht mag.}

    \begin{answerbox}[1in]
    \end{answerbox}
\end{exercise}

\mandala{mandala/mandala01}

\end{document}
