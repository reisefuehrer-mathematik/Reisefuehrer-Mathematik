\documentclass[../main.tex]{subfiles}

\begin{document}

\if 0
\textbf{Inhalte des Vorwortes:}
\begin{itemize}
    \item Für wen ist dieses Buch? -- Das Buch ist kostenlos und für jeden, hey! :D
    \item Link zum Internetauftritt, falls vorhanden
    \item Nutzung: Nachschlagewerk, Selbstlernen, Flipped Classroom
    \item Vorstellen unserer Philosophie (Beispiele erst! -> Zitat: Def zuerst ist scheiße)
    \item Erklärung der Buchelemente (Satz, Definition, Zusatzwissen, Nutshell...)
    \item Erläuterung von Flipped Classroom
    \item Sehr großer Dank an alle Autoren und namentliche Nennung mit Bild, falls gewünscht
    \item Dank an alle Korrektoren
    \item Dank an Arne? Weil KVA so toll ist?
\end{itemize}
\fi

Die Mathematik ist ein Fachgebiet, in dem man versucht, ausgehend von einfachen Annahmen (Axiomen) eine Art Werkzeugkoffer mit Hilfsmitteln zu erschaffen, die uns das Leben vereinfachen. Darüber hinaus wirft die Mathematik mitunter sehr interessante und tiefgehende Fragen auf und führt hin und wieder zu sehr verblüffenden Erkenntnissen. Das Schöne daran ist, dass wir dafür außer unserem Verstand keine weiteren Hilfsmittel benötigen. Alles, was man in der Mathematik lernt, baut logisch aufeinander auf, sodass ein riesiges und beeindruckendes Kunstwerk entsteht, das mit einem sehr kleinen, einfachen Fundament auskommt.

Gerade weil die Mathematik ein so beeindruckendes und faszinierendes Fachgebiet ist, verdient sie es auch, vernünftig erklärt zu werden. Wenn wir einem Schüler verbieten, seine Lieblingslieder mitzusingen, bis er gelernt hat, Noten zu lesen und den Quintenzirkel auswendig kennt, dann wird er vermutlich die Freude an der Musik verlieren, obwohl er vielleicht sogar das Zeug zum Opernsänger gehabt hätte.

In unserem Buch bilden wir den gerade beschriebenen Aufbau der Mathematik ab, indem wir versuchen, alle Themen lückenlos und logisch zu erklären. Unser Ziel ist es, zu vermeiden, dass während des Buchs plötzlich überraschende Behauptungen \enquote{vom Himmel fallen}, weil wir überzeugt davon sind, dass es sonst nicht richtig möglich ist, die Mathematik zu verstehen.

Dieses Buch behandelt alle Themen, die in der Schule ab der fünften Klasse vorkommen sowie ein paar weitere Themen, die aus unserer Sicht notwendig waren, um das gerade beschriebene Fundament zu errichten. An manchen Stellen vereinfachen wir uns das Leben, indem wir auf eine mathematisch exakte Darstellung verzichten, wenn sie für das Verständnis nicht unbedingt notwendig ist. Damit unser Buch lückenlos ist, sind außerdem auch hin und wieder komplizierte Argumente notwendig, die eher störend sind, wenn es nur darum geht, das eigentliche Thema zu verstehen. In diesen Fällen präsentieren wir zwar die vollständige Herleitung, markieren sie aber als weiterführendes Wissen.

Um ein Fach am Ende so zu vermitteln, dass man es gut verstehen kann, wird neben einem Lehrbuch selbstverständlich auch ein gutes Lehrkonzept benötigt. Klassische Lehre sieht in der Regel so aus, dass ein Lehrer versucht, einer größeren Gruppe von Schülern ein gewisses Thema zu erklären, das er in einem Tempo, das er selbst wählt, präsentiert. Beim Vermitteln von neuem Stoff ist es auf diese Weise nicht möglich, auf die Lerngeschwindigkeit aller Schüler gleichzeitig einzugehen. Wählt man ein langsameres Tempo, damit alle mitkommen, unterfordert man zwangsläufig einige Schüler. Andersrum gilt das natürlich genauso.

Nicht immer verstehen alle Schüler direkt auf Anhieb, was der Lehrer versucht hat, ihnen zu erklären. Den Stoff zu Hause allein nachzuarbeiten, ist mit den Materialien, die die Schule zur Verfügung stellt, jedoch kaum möglich: In der Regel bekommt man vor allem Aufgaben mit nach Hause, die jedoch nichts bringen, wenn man die Grundlagen für sie nicht verstanden hat. In diesem Fall würde eine gute Erklärung helfen, die zu Hause in Ruhe noch einmal durchgegangen werden kann.

Im traditionellen Frontalunterricht gibt es also zwei Probleme: Im Unterricht (während die Schüler Zugriff auf den Lehrer und Mitschüler haben), werden die meisten Schüler über- oder unterfordert, weil ihnen ein festes Tempo vorgegeben wird.
Wenn dann beim Bearbeiten der Hausaufgaben Fragen auftreten, dann hat man als Schüler verloren, weil es niemanden gibt, den man wirklich fragen kann.

Das Lehrkonzept \emph{Flipped Classroom} versucht, diese Probleme zu lösen, indem diese Konstellation umgedreht wird: Zuhause hat der Schüler die Aufgabe, sich mit neuem Stoff zu beschäftigen. Er muss nicht alles verstehen, aber sollte ehrlich mit sich selbst sein und sich das Material gewissenhaft in seinem eigenen Tempo anschauen. Im Klassenzimmer können die Schüler dann Fragen stellen und den Stoff anschließend mit Übungsaufgaben festigen (während die Schüler Zugriff auf den Lehrer und Mitschüler haben).

Damit das Konzept \emph{Flipped Classroom} erfolgreich umgesetzt werden kann, benötigen die Schüler Zugriff auf Lehrmaterial, mit dem sie den Stoff sinnvoll zu Hause lernen können. Für den Erfolg des Konzepts ist die Qualität dieses Materials zentral wichtig. Kurze, formale Erklärungen, die eigentlich an Universitäten gehören, werden die wenigsten Schüler ans Ziel führen -- ebenso wenig wie eine große Fülle an Übungsaufgaben ohne ausreichende Erklärungen, wie an diese heranzugehen ist. Was es stattdessen braucht, sind also anschauliche Erklärungen anhand von Beispielen, die es ermöglichen, sich den Stoff tatsächlich selbst anzuschauen und dabei zumindest grob zu verstehen. Mögliche Fragen, die übrig bleiben, können dann immer noch im Unterricht geklärt werden.

Dieses Buch unterscheidet sich von den meisten Lehrbüchern, weil es keine Aufgaben enthält. Den Platz, den wir dadurch gewonnen haben, dass wir die Aufgaben weggelassen haben, haben wir stattdessen mit Beispielen und ausführlicheren Erklärungen gefüllt. Auf diese Weise ist dieses Buch ein Teil des Lehrkonzeptes \enquote{Reiseführer Mathematik}, das neben diesem Buch noch eine Reihe weiterer Lehrmaterialien umfasst, die genutzt werden können, um \emph{Flipped Classroom} effizient umzusetzen.

Das Buch wird ergänzt durch bestimmte Arbeitsblätter, die wir \emph{Vorbereitungsblätter} nennen. Sie sind eine Art Anleitung für die Arbeit mit dem Buch und beziehen sich immer auf einen bestimmten Abschnitt des Buchs. Auf diesen Vorbereitungsblättern erhalten Schüler somit die Aufgabe, was sie sich zu Hause anschauen sollen: Das Vorbereitungsblatt verweist auf Stellen im Buch und stellt darüber hinaus über QR-Codes passende Lehrvideos zur Verfügung, die genau die gleichen Inhalte wie die Buchstellen abdecken. Auf diese Weise haben Schüler die freie Wahl, ob sie sich lieber ein kurzes Video anschauen oder dasselbe Thema im Buch nachlesen. In jedem Fall steht ihnen das Buch aber als Nachschlagewerk zur Verfügung, sodass es nicht erforderlich ist, jedes Mal erneut ein Video zu schauen, wenn nur ein bestimmtes Detail daraus unklar ist.

Um zu überprüfen, ob der Stoff einigermaßen verstanden wurde, können die kurzen Verständnisaufgaben genutzt werden, die ebenfalls auf den Vorbereitungsblättern zu finden sind. Wenn Fragen entstehen, können diese dort notiert und zu Beginn der nächsten Unterrichtsstunde geklärt werden. Dafür werden am Anfang der Stunde Fragen gesammelt und das Vorbereitungsblatt besprochen. Als weitere Hilfestellung kann der Lehrer ein kleines Quiz vorbereiten oder mit einem kurzen, interaktiven Spiel den Stoff festigen.

Anschließend kann die weitere Zeit genutzt werden, um Übungsaufgaben zu beantworten. Statt alle Schüler die gleichen Aufgaben rechnen zu lassen, kann sich jeder Schüler Aufgaben aus einem \emph{Aufgabenblock} aussuchen, die zu seinem Lernstand passen. Einen solchen Aufgabenblock gibt es zu jedem Kapitel dieses Buchs. Die Aufgaben sind dafür mit Schwierigkeitsgraden markiert. Unter anderem gibt es auch Knobelaufgaben, die speziell für solche Schüler gestaltet sind, denen es nichts bringen würde, die gewöhnlichen Aufgaben zu bearbeiten, weil sie zu einfach sind.

An der Erstellung dieses Lehrmaterials hat ein großes Team aus Schülern, Lehrern und Studenten in ihrer Freizeit ehrenamtlich mitgewirkt und tausende von Stunden investiert. Über einen Zeitraum von mehreren Jahren ist dabei das Ergebnis entstanden, das du gerade liest. 
In dieses Buch eingeflossen sind zahlreiche Unterrichtserfahrungen von Lehrern, die sich bereit erklärt haben, unser Lehrmaterial auszuprobieren und mit den Rückmeldungen stark zur Verbesserung unseres Materials beigetragen haben. Unser Dank dafür gilt den Lehrern [Lehrer erwähnen].

Darüber hinaus enthält dieses Buch das intuitive Verständnis einer ganzen Reihe von Mathematikstudenten, die dieses genutzt haben, um über 1000 Seiten von anschaulichen Erklärungen zu verfassen, die Schülern einen einfacheren Zugang zur Mathematik ermöglichen sollen. 
Für die vielen geopferten Stunden Freizeit, die in das Verfassen dieser Seiten geflossen sind, können wir uns überhaupt nicht genug bei all denjenigen bedanken, die über Jahre hinweg an unserem Buch, den Aufgaben, den Musterlösungen oder den Videokonzepten mitgeschrieben haben: Robin Schmöcker, Tim Hagen, [Liste erweitern].

In diesem Buch und unseren anderen Materialien tauchen immer wieder wunderschöne kleine Zeichnungen auf, die unsere Erklärungen mit Leben füllen. Für das Erstellen dieser Zeichnungen möchten wir uns bei unseren Illustratoren [Liste] bedanken.

Schließlich möchten wir uns bei [] für ihre Unterstützung und bei [] für ihre wertvollen Anmerkungen bedanken.

Lehrte, im Mai 2022 \hfill Tobias Brockmeyer (Gründer des Projekts)

\if 0
Leider sind Lehrer nicht immer die besten Lehrer.

Wer kennt das nicht: bei der ausgebildeten Lehrkraft ist es sehr schwer, dem Unterricht zu folgen und mitzudenken, wohingegen es der Quereinsteiger spaßig und verständlich erklärt. Natürlich ist das keine allgemeingültige Wahrheit. Es gibt auch schwarze Schafe unter den Quereinsteigern und leuchtende Musterbeispiele unter den Lehrern. Leider können derartige Unterschiede in der Qualität des Unterrichts weitreichende Folgen haben. Für 2 Schuljahre hat man einen Lehrer, mit dem man keine Wellenlänge teilt und bei dem man den Stoff nicht versteht. Nun hat er ein möglichst einfaches Fach auf Lehramt studiert, und man darf es ausbaden weil man komplett den Anschluss verloren hat.

Das Ziel dieses Werkes ist zwiefältig. Wir möchten Schülerinnen und Schülern ein \emph{kostenloses} und \emph{frei verfügbares} Nachschlagewerk sowie Übungsmaterial bieten, um die eigenen Schwächen mit der Mathematik möglichst einfach Nachzuarbeiten. Zudem möchten wir das Konzept des \enquote{\emph{Flipped Classroom}} für Lehrer attraktiv machen.

Flipped Classroom erkennt zwei große Probleme im traditionellen Frontalunterricht: im Unterricht (während die Schüler Zugriff auf den Lehrer und Mitschüler haben) wird still gesessen und wenn der Schüler den Stoff nicht verstanden hat, ist er für die Hausaufgabe \enquote{gearscht}, weil er niemanden wirklich fragen kann.
Dieses Problem wird durch Flipped Classroom gelöst, indem diese Konstellation umgedreht wird: zu Hause hat der Schüler die Aufgabe, sich mit dem Material zu beschäftigen -- er muss nicht alles verstehen aber sollte ehrlich mit sich selbst sein, und sich das Material gewissenhaft anschauen -- im Klassenzimmer können die Schüler dann Fragen stellen und üben den Stoff. \textbf{Zu Hause wird der Stoff (gewissenhaft) vorbereitet und im Klassenzimmer wird er geübt und damit gelernt.}

Wenn du dieses Buch liest, wird dir auffallen, dass der Fließtext immer wieder durch verschiedene, farbig hervorgehobene, Boxen unterbrochen wird. Insbesondere die \enquote{Kurz und knapp} Boxen sind dabei wahrscheinlich interessant. Wenn du ein Unterkapitel bereits verstanden hast, dann solltest du nicht alles nochmal lesen müssen, um es wieder aufzuarbeiten. Die \enquote{Kurz und knapp} Boxen geben dir eine kurze Zusammenfassung der wichtigsten Erkenntnisse des jeweiligen Unterkapitels. Hast du den Schulstoff so weit verstanden und möchtest mehr erfahren oder bist du einfach neugierig? Dann kannst du dir die \enquote{Tiefer in den Hasenbau} Boxen durchlesen. Hier behandelte Themen sind nicht relevant für den Rest des Buches aber sollen interessierten Schülern erlauben, sich selbst herauszufordern.

Bleibt zuletzt noch die Danksagung, die dem Leser nichts bedeutet aber für uns ein wichtiges Zeichen unserer Wertschätzung darstellt.
Dank gebührt allen Autoren, die ihre Freizeit aufgebracht haben, um sich viele Gedanken zu machen und viel Mühe in dieses Lehrbuch und Übungskonzept zu stecken. Ebenso bedanken wir uns bei allen Lehrern und Korrektoren. Insbesondere möchten wir uns bei Arne bedanken, der uns das Konzept des Flipped Classroom gepaart mit minimaler Hilfe in Aktion gezeigt hat. Der größte Dank gebührt aber selbstredend den Katzen der Autoren, ohne die es schlicht nicht möglich gewesen wäre, sich für ein solches Projekt zu motivieren.

{\hfill \Large Gutes Gelingen!}

\bigskip
% Ich hätte nichts dagegen, wenn wir dieses Zitat tatsächlich einfach so lassen
\epigraph{\enquote{Hier kommt ein echt tolles Zitat hin. Es wird ein großartiges Zitat sein. Das beste Zitat, das je jemand gewählt hat. Hoffentlich vergessen wir nicht, hier etwas einzufügen.}}{-- Jemand Wichtiges}
\fi

\end{document}