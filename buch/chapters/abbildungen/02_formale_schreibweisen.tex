\documentclass[../../main.tex]{subfiles}

\begin{document}
\label{sec:abbildungen_definition}
Nun sind wir bereit für die formale Definition von Abbildungen. Um besser über Dinge reden zu können, möchte man ihnen gern Namen geben. So könnte man der Abbildung aus den vorherigen Beispielen den Namen \textsc{EiskugelPreis} geben. Damit ist es leichter, später weiter über diese Abbildung zu sprechen -- man hat nun einen bestimmten Namen, mit dem man genau diese Abbildung verbindet.

Entsprechend soll auch formal jede Abbildung einen bestimmten Namen erhalten. Als Formelsymbol für Abbildungen wird hier ein $f$ verwendet. Dies kommt vom englischen Wort \emph{function}. In dieser Definition werden einige Schreibweisen eingeführt, die anschließend weiter diskutiert werden.

\begin{definition}[Abbildung]
Eine \textbf{Abbildung} $f$ ist eine Vorschrift, die allen Elementen aus der Definitionsmenge $U$ genau ein Element aus der Zielmenge $V$ zuordnet, geschrieben $f\colon U\rightarrow V$. Wird ein Element $u\in U$ auf $v\in V$ abgebildet, so schreibt man $f(u)=v$ bzw. $u\mapsto v$.
\end{definition}

Im Großen und Ganzen ist dies eine Wiederholung von dem, was wir soeben gesehen haben. Neu sind im Wesentlichen zwei Schreibweisen: 

Die erste Schreibweise ist \mbox{$f\colon U\rightarrow V$}. Sie sagt zunächst einmal aus, dass wir eine Abbildung mit dem Namen $f$ haben. Nach dem Doppelpunkt erklären wir dann, was für eine Abbildung $f$ ist, d.h. von welchen Werten wir erwarten können, dass $f$ ihnen ein Bild zuordnet (das macht $f$ genau für die Elemente aus $U$) und welche möglichen Bilder wir erwarten können (alle Bilder müssen aus der Zielmenge $V$ stammen). Man könnte sie also lesen als \enquote{$f$ ist eine Abbildung von $U$ nach $V$}.

\begin{example}{}
    Um auszudrücken, dass das Eis-Preisschild die Abbildung \textsc{EiskugelPreis} darstellt, die Eiskugelgrößen, also Elementen der Menge \textsc{Kugeln}, einen Preis, also ein Element aus \textsc{MöglichePreise}, zuordnet, kann man \[\textsc{EiskugelPreis}\colon\textsc{Kugeln}\rightarrow\textsc{MöglichePreise}\] schreiben. Man meint damit also, dass wir eine Abbildung namens \textsc{EiskugelPreis} haben, deren Definitionsmenge die Menge \textsc{Kugeln} ist und die Elemente aus dieser Definitionsmenge auf \textsc{MöglichePreise} abbildet.
\end{example}

Die zweite Schreibweise ermöglicht es, nachdem man klargestellt hat, welche Art von Objekten eine Abbildung übersetzt (d.h. nachdem man Definitions- und Zielmenge festgelegt hat), festzulegen, \emph{wie} die Abbildung diese Übersetzung vornimmt. Dazu lässt sich mit der Schreibweise $u\mapsto v$ formulieren, dass $u$ das Bild $v$ hat, d.h. die Abbildung übersetzt das Element $u$ aus der Definitionsmenge in das Element $v$ aus der Zielmenge.

Üblicher ist es, genau dieselbe Information durch die Schreibweise $f(u)=v$ (gelesen: \emph{$f$ von $u$ ist $v$}) auszudrücken. Damit gibt man nämlich direkt an, \emph{welche} Abbildung $u$ zu $v$ übersetzt.

\begin{example}{}
    \parpic[r]{
        \centering
        \begin{tikzpicture}[scale=.6]
            \node at (0,0) {\includegraphics[width=.3\textwidth]{images/dictionary.png}};
            \node at (-1.6,1.5) {\small\textbf{Deutsch}};
            \node at (1.6,1.5) {\small\textbf{English}};
            \node at (1.6,0) {\small\textit{dog}};
            \node at (-1.6,0) {\small\textit{Hund}};
            \node at (1.6,-1) {\small\textit{cat}};
            \node at (-1.6,-1) {\small\textit{Katze}};
        \end{tikzpicture}
    }
    In verschiedenen Sprachen, zum Beispiel Deutsch und Englisch, verwendet man unterschiedliche Wörter für die gleichen Dinge. Im Englischunterricht lernt man deshalb Vokabeln, etwa, dass das englische Wort für \emph{Hund} das Wort \emph{dog} ist. Hier versteckt sich erneut eine Abbildung. Diese könnte man zum Beispiel \textsc{EnglischesWort} nennen und sie würde aus dem deutschen Wort \emph{Hund} das englische Wort \emph{dog} machen. Entsprechend schreibt man
    \[\textit{Hund}\mapsto\textit{dog}.\]
    Ebenfalls wäre es möglich gewesen, $\textsc{EnglischesWort}(\textit{Hund})=\textit{dog}$ zu schreiben. Damit drückt man genau dasselbe aus und auch ausgesprochen als \enquote{\emph{\textsc{EnglischesWort} von Hund ist dog}} ist klar, dass dies richtig sein muss.
\end{example}

Außerdem lohnt es sich, an dieser Stelle zu betonen, dass eine Abbildung nicht unbedingt jedes Element aus der Zielmenge verwenden muss, d.h. nicht jedes Element der Zielmenge muss ein Urbild haben. Stattdessen ist die Zielmenge vielmehr ein Vorrat an möglichen Bildern, der ausgeschöpft werden kann, aber nicht muss.

Wie diese Schreibweisen nun bei weiteren konkreten Abbildungen aussehen, schauen wir uns zunächst nochmal am Beispiel mit der Eisdiele an. Anschließend sehen wir ein weiteres Beispiel aus dem Leben, das sich wie eine Abbildung verhält und verwenden auch dort die gerade eingeführten Schreibweisen.

\begin{example}{}
    \parpic[r]{
        \begin{tikzpicture}[scale=.75]
            \fill[grayset] (-1.5,0) ellipse (0.5cm and 2cm);
            \fill[grayset] (3.5,0) ellipse (1.5cm and 2cm);
            %elements
            \node[blue!70] (x1) at (-1.5,0.7) {\LARGE $\bullet$};
            \node[orange] (x2) at (-1.5,-0.7) {$\bullet$};
            \node (y1) at (3.5,0.7) {1.50 \euro};
            \node (y2) at (3.5,-0.7) {1.00 \euro};
            %arrows
            \draw[->] (x1) -- node[above] {\textcolor{blue!70}{$\bullet$}$\mapsto$\small 1.50 \euro} (y1);
            \draw[->] (x2) -- node[above] {\textcolor{orange}{$\bullet$}$\mapsto$\small 1.00 \euro} (y2);
        \end{tikzpicture}
    }
    Das Preisschild an der Eisdiele besagt, dass \emph{kleine Kugeln} einen Preis von 1.00~\euro{} haben und dass \emph{große Kugeln} 1.50~\euro{} kosten. 
    
    \picskip{4}
    Aus mathematischer Sicht wäre es vollkommen ausreichend, die Abbildung als \[\textsc{EiskugelPreis}\colon\textsc{Kugeln}\rightarrow\textsc{MöglichePreise}\] 
    mit der Regel zu beschreiben, dass $\textsc{EiskugelPreis}(\textit{kleine Kugel})=\text{1.00 \euro}$ gilt, also eine kleine Kugel 1.00~\euro{} kostet und $\textsc{EiskugelPreis}(\textit{große Kugel})=\text{1.50 \euro}$, d.h. eine große Kugel kostet 1.50~\euro. 
    
    Diese letzten beiden Regeln entscheiden also, wie wir in einem Diagramm die Pfeile einzeichnen müssen. Mit anderen Worten legen wir hier einfach nur mit der formalen Schreibweise fest, welche Kugelgröße wie viel kostet.
\end{example}

\begin{example}{}
    Durch das Mischen der drei Grundfarben rot, blau und gelb kann man beliebige Farben erhalten. Mischt man beispielsweise rot mit gelb, so erhält man orange. Das folgende Diagramm zeigt, was mit den drei Grundfarben passiert, wenn man sie mit gelb mischt: Sie werden zu grün, orange bzw. gelb.
    \begin{center}
        \begin{tikzpicture}[scale=.6]
    \fill[grayset] (-1.5,0) ellipse (2.4cm and 2cm);
    \fill[grayset] (4,0) ellipse (2.4cm and 2cm);
    %
    \node[label={[blue]left:blau}, blue] (x1) at (-1,0.7) {$\bullet$};
    \node[label={[red]left:rot}, red] (x2) at (-1,-0.2) {$\bullet$};
    \node[label={[yellow!70!black]left:gelb}, yellow!70!black] (x3) at (-1,-1.1) {$\bullet$};
    %
    \node[label={[green!70!black]right:grün}, green!70!black] (y1) at (3.5,0.7) {$\bullet$};
    \node[label={[orange]right:orange}, orange] (y2) at (3.5,-0.2) {$\bullet$};
    \node[label={[yellow!70!black]right:gelb}, yellow!70!black] (y3) at (3.5,-1.1) {$\bullet$};
    %
    \draw[->] (x1) to[bend left] (y1);
    \draw[->] (x2) -- (y2);
    \draw[->] (x3) to[bend right] (y3);
\end{tikzpicture}%also used by abbildungen/03_abbildungen_beschreiben
    \end{center}
    Dieser Vorgang beschreibt eine Abbildung \textsc{MitGelbMischen}. Zu Beginn kann man eine beliebige Grundfarbe haben, also ist die Definitionsmenge der Abbildung die Menge der Grundfarben -- oder mathematisch \[\textsc{Grundfarben}\coloneqq\{\text{blau},\text{rot},\text{gelb}\}.\]
    Nachdem man eine Grundfarbe mit gelb gemischt hat, bekommt man eine Mischfarbe. Die drei Mischfarben, die wir hier erhalten können, sind die Zielmenge der Abbildung \textsc{MitGelbMischen}. Formal ist die Zielmenge demnach die Menge \[\textsc{Mischfarben}\coloneqq\{\text{grün},\text{orange},\text{gelb}\}.\]
    Grundfarben mit der Farbe gelb zu mischen, erklärt also eine Abbildung \[\textsc{MitGelbMischen}\colon\textsc{Grundfarben}\rightarrow\textsc{Mischfarben}\] mit der Abbildungsvorschrift $\text{blau}\mapsto\text{grün}$, $\text{rot}\mapsto\text{orange}$ und $\text{gelb}\mapsto\text{gelb}$. Alternativ kann man auch die Schreibweise nutzen, bei der man das Argument in Klammern setzt. Dieselbe Abbildungsvorschrift kann man dann durch $\textsc{MitGelbMischen}(\text{blau})=\text{grün}$, $\textsc{MitGelbMischen}(\text{rot})=\text{orange}$ und $\textsc{MitGelbMischen}(\text{gelb})=\text{gelb}$ notieren. Es lohnt sich, noch einmal zu betonen, dass diese beiden Schreibweisen tatsächlich dasselbe meinen.
\end{example}

\begin{nutshell}{Grundbegriffe zu Abbildungen}
    \parpic[r]{
        %\begin{wrapfigure}{r}{.3\textwidth}
    %\centering
    \begin{tikzpicture}[scale=.6]
        \draw[grayset] (-1.5,0) ellipse (0.7cm and 2cm);
        \draw[grayset] (1.5,0) ellipse (0.7cm and 2cm);
        %
        \node at (-1.5,1.5) {$U$};
        \node at (1.5,1.5) {$V$};
        \node at (0,0.2) {$f$};
        %
        \node (x1) at (-1.5,0.7) {$\bullet$};
        \node (x2) at (-1.5,-0.2) {$\bullet$};
        \node (x3) at (-1.5,-1.1) {$\bullet$};
        \node (y1) at (1.5,0.7) {$\bullet$};
        \node (y2) at (1.5,-0.2) {$\bullet$};
        \node (y3) at (1.5,-1.2) {$\bullet$};
        %
        \draw[->] (x1) -- (y3);
        \draw[->] (x2) to[bend right] (y1);
        \draw[->] (x3) to[bend right] (y2);
    \end{tikzpicture}
%\end{wrapfigure}%Needed by tasks
    }
    Abbildungen sind in der Mathematik eine Regel, mit der Elemente von einer Menge, der \textbf{Definitionsmenge}, in Elemente der \textbf{Zielmenge} übersetzt werden.
    
    \picskip{2}
    Eine Abbildung mit dem Namen $f$, die aus einer Definitionsmenge $U$ in eine Zielmenge $V$ abbildet, schreibt man als $f\colon U\rightarrow V$ (gelesen: $f$ ist eine Abbildung von $U$ nach $V$).
    
    Um eine Abbildung anzuwenden, nimmt man ein Element der Definitionsmenge (das \textbf{Argument} der Abbildung) und erhält dessen \textbf{Bild}, also ein Element der Zielmenge, das die Abbildung eindeutig festlegt. 
    
    Als Formel schreibt man, wenn die Abbildung $f$ das Argument $u\in U$ auf das Element $v\in V$ abbildet, $f(u)=v$. Alternativ kann man auch $u\mapsto v$ schreiben (d.h. $u$ wird auf $v$ abgebildet), um explizit zu betonen, dass $u$ das \textbf{Bild} $v$ hat bzw. $u$ ein \textbf{Urbild} von $v$ ist.
\end{nutshell}

\end{document}