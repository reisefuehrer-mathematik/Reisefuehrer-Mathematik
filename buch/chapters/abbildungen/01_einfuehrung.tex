\documentclass[../../main.tex]{subfiles}

\begin{document}
\label{sec:abbildungen_intuition}
Aus den letzten Kapiteln sollte dir bereits bekannt sein, dass Mengen in der Mathematik dazu dienen, über eine bestimmte Auswahl von Objekten -- ganz egal, ob es Zahlen, Farben, Gegenstände oder andere Dinge sind -- zu reden. Anschaulich konnte man sich Mengen durch sogenannte Venn-Diagramme vorstellen. Beispielsweise siehst du ein solches Diagramm für eine einzelne Menge in der folgenden Abbildung.

\begin{figure}[ht]
    \centering
    
    \begin{tikzpicture}
        \fill[grayset] (-1.5,0) circle (15mm);
        \node[label={[red]above:rot}, red] at (-1.5,0.7) {\textbullet};
        \node[label={[blue]above:blau}, blue] at (-2.5,0) {\textbullet};
        \node[label={[yellow!70!black]above:gelb}, yellow!70!black] at (-1.5,-1.1) {\textbullet};
        \node[label={[green!70!black]below:grün}, green!70!black] at (1.3,0.3) {\textbullet};
        
        \node (setName) at (1.2,1.5) {\textsc{Grundfarben}};
        \draw[->] (setName) to[bend left] (-0.3,0.5);
    \end{tikzpicture}
    
    \caption{Diese schematische Abbildung stellt die Menge der Grundfarben, wie wir sie bereits kennen, dar. Graphisch wird die Menge durch einen Kreis repräsentiert. Objekte, die zur Menge gehören, also die \textbf{Elemente} der Menge, sind im Kreis eingezeichnet, während z. B. die Farbe \emph{grün}, die keine Grundfarbe ist, nicht zur Menge gehört und deswegen auch nicht im Kreis abgebildet ist.}
    \label{fig:venn_diagram}
\end{figure}

Neben einer praktischen Möglichkeit, Sammlungen von Objekten mathematisch darzustellen, sind Mengen auch die Grundlage für viele weitere Konstrukte in der Mathematik. Den wohl wichtigsten Begriff, der auf Mengen aufbaut, nämlich den der \textbf{Abbildung} bzw. \textbf{Funktion} lernen wir in diesem Kapitel kennen.

Abbildungen sind aus vielerlei Hinsicht relevant: Erst einmal erlauben sie es uns, später über nochmals andere mathematische Objekte zu sprechen. Darüber hinaus funktioniert beispielsweise auch moderne künstliche Intelligenz auf Basis von Abbildungen. Durch Abbildungen ist es Computern also möglich, das Lösen von komplizierten Aufgaben zu lernen -- zum Beispiel das Erkennen von handschriftlichem Text.

Bevor wir uns Abbildungen genauer ansehen und mathematisch analysieren, werfen wir zunächst einen Blick auf einige Beispiele, die uns im Leben begegnen und die sich wie Abbildungen verhalten.

\begin{example}
    \parpic[r]{
        \begin{tikzpicture}[scale=.6]
            \node at (0,0) {\includegraphics[width=.3\textwidth]{images/icecream.png}};
            \node at (1.5,1.7) {\small\textbf{pro Kugel}};
            \node[label=above:1.00, orange] at (0.6,0) {\large $\bullet$};
            \node[label=above:1.50, blue!60] at (2.4,0) {\LARGE $\bullet$};
        \end{tikzpicture}
    }
    Der Besitzer einer Eisdiele, der Eiskugeln verschiedener Größe verkauft, möchte seine Kunden darauf hinweisen, wie viel eine Kugel Eis bei ihm kostet, um erstens Verwirrungen zu vermeiden und zweitens damit zu werben, dass Eis bei ihm billiger als anderswo ist.
    
    \picskip{3}Deswegen hängt er das nebenstehend abgebildete Preisschild vor die Tür, auf dem er schreibt, dass er kleine Kugeln für 1.00\,\euro{} verkauft und große Kugeln für 1.50\,\euro. Außerdem kostet extra Sahne auf dem Eis bei ihm 50 Cent (das steht allerdings aus Platzgründen nicht auf der Tafel).
    
    Tatsächlich genügt dieses kleine Beispiel schon, um eine Abbildung wiederzufinden. Das Preisschild, das der Inhaber vor seine Tür gestellt hat, macht nämlich aus der Information, welche Kugel man kaufen möchte, den Preis dieser Kugel.
    
    Ausgehend von entweder einer kleinen oder einer großen Kugel erhält man also den Preis dieser Kugel. Dabei hängt der Preis natürlich davon ab, welche Größe man haben möchte. Das Schild macht daher aus \emph{einer bestimmten} Größe \emph{einen bestimmten} Preis. Man kann auch sagen, das Schild \textbf{bildet} eine Eiskugelgröße auf ihren Preis \textbf{ab}. 
    
    Wichtig hierbei ist, dass jede Eiskugel im Angebot auch einen Preis erhält -- und genau einen, denn eine Kugel, die mehrere Preise gleichzeitig hat, ergibt wenig Sinn. Diese Eigenschaft findet sich auch bei Abbildungen in der Mathematik wieder, wie wir später sehen werden.
\end{example}

Grundsätzlich ist der Sinn einer Abbildung in der Mathematik, von einer Menge in eine andere abzubilden. Das sind in der Regel Mengen von Zahlen, doch es spricht prinzipiell nichts dagegen, stattdessen Eiskugelgrößen, Farben oder andere Dinge als Mengen zu verwenden. Genau genommen nimmt die Abbildung immer ein bestimmtes Element aus der ersten Menge und übersetzt es in ein Element der zweiten Menge. 

Praktischerweise gibt es hierfür ebenso wie für Mengen eine sehr übersichtliche Möglichkeit, diesen Sachverhalt anschaulich darzustellen. Zunächst einmal muss man sich überlegen, dass Abbildungen ganz bestimmte Einsatzgebiete haben: Jede Abbildung bildet Elemente einer ersten Menge, der \textbf{Definitionsmenge}, auf andere Elemente ab. Für Objekte, die nicht aus der Definitionsmenge stammen, ergibt die Abbildung dann meistens keinen Sinn -- oder sie ist für diese schlicht nicht definiert.

Bekommt eine Abbildung nun ein Element aus der Definitionsmenge, so übersetzt sie es in ein anderes Element. Auch hier kommt wieder nicht irgendetwas heraus, sondern es lässt sich eingrenzen, welche Abbilde die Abbildung produziert. Die Menge der möglichen Bilder nennt man \textbf{Zielmenge}.

Damit bietet es sich an, als Veranschaulichung die Definitions- und die Zielmenge nebeneinander zu zeichnen. Anschließend verbindet man mit Pfeilen die Elemente der Definitionsmenge mit ihren jeweiligen Abbilden.

\begin{figure}[ht]
    \centering
    \begin{tikzpicture}[scale=.75]
    \draw[grayset] (-1.5,0) ellipse (0.7cm and 2cm);
    \draw[grayset] (1.5,0) ellipse (0.7cm and 2cm);

    \node (x1) at (-1.5,0.7) {$\bullet$};
    \node (x2) at (-1.5,-0.2) {$\bullet$};
    \node (x3) at (-1.5,-1.1) {$\bullet$};
    \node (y1) at (1.5,0.7) {$\bullet$};
    \node (y2) at (1.5,-0.2) {$\bullet$};
    \node (y3) at (1.5,-1.2) {$\bullet$};

    \draw[->] (x1) -- (y3);
    \draw[->] (x2) to[bend right] (y1);
    \draw[->] (x3) to[bend right] (y2);

    \draw[maincolor, <-] (-1.1,1) to[bend right] (-0.2,1.7);
    \draw[maincolor, <-] (1.2,-1.5) to[bend right] (0.3,-1.85);

    \node[maincolor, text width=15mm, align=center] at (0,-0.8) {\tiny Funktion};

    \node[maincolor, text width=15mm, align=center] at (-0.2,1.77) {\scriptsize menge};
    \node[maincolor, text width=15mm, align=center] at (-0.2,1.85) {\scriptsize Definitions-};

    \node[maincolor, text width=15mm, align=center] at (0.3,-1.98) {\scriptsize Zielmenge};;
\end{tikzpicture}%Also used by abbildungen/04_abbildungen_in_koordinatensystemen
    
    \caption{In dieser Grafik ist alles zu sehen, was für eine Abbildung relevant ist: Links zu sehen ist die \textbf{Definitionsmenge}, für deren Elemente die Abbildung zur Anwendung kommt. Auf der rechten Seite abgebildet ist die \textbf{Zielmenge} -- die Menge, aus der alle Ergebnisse nach Anwendung der Abbildung kommen. Die Pfeile stellen schließlich dar, auf welche Weise die Abbildung zwischen Definitions- und Zielmenge abbildet.}
    \label{fig:function_scheme}
\end{figure}

\begin{example}
    \parpic[r]{
        \begin{tikzpicture}[scale=0.8]
            \draw[grayset] (-1.5,0) circle[radius=15mm];
            \node (setName) at (0,2) {\textsc{Definitionsmenge}};
            \node[label={[blue!70]below:groß}, blue!60] (x1) at (-1.1,0.9) {\LARGE $\bullet$};
            \node[label={[orange]below:klein}, orange] (x2) at (-1.8,-0.4) {$\bullet$};
            \node (x3) at (-1.5,-1.85) {$\bullet$};
            \node[label=above:1.50 \euro] (y1) at (1.2,0.6) {$\bullet$};
            \node[label=above:1.00 \euro] (y2) at (1.2,-0.6) {$\bullet$};
            \node[red] (y3) at (1.2,-1.6) {\textbf{?}};
            \draw[->] (setName) to[bend right] (-2.7,0.5);
            \draw[blue,->] (x1) -- (y1);
            \draw[orange,->] (x2) -- (y2);
            \draw[->] (x3) -- (y3);
        \end{tikzpicture}
    }
    Im obigen Beispiel ist die \textbf{Definitionsmenge} die Menge bestehend aus den Größen \emph{groß} und \emph{klein}, also 
    \[\textsc{Kugeln}\coloneqq\{\text{große Kugel},\text{kleine Kugel}\}.\]
    Während es sinnvoll ist, zu fragen, wie viel eine große oder kleine Kugel kostet, wird das Schild beispielsweise keine Auskunft darüber geben, wie teuer ein Autoreifen, ein Notizblock oder eine Kartoffel ist -- ganz einfach, weil das Schild vor einer Eisdiele steht und ausschließlich dafür gemacht ist, den Preis von Eis anzugeben.
\end{example}

Natürlich benötigt man in der Mathematik eine ordentliche Schreibweise für Abbildungen. Bevor wir diese einführen, klären wir noch ein paar weitere wichtige Begriffe. Wenn eine Abbildung ein Element $x$ aus der Definitionsmenge bekommt und in ein anderes übersetzt, dann nennt man das neu erhaltene Element das \textbf{Bild} von $x$.

Geht man andersherum von einem Element $y$ aus der Zielmenge aus und findet ein Element, das zu diesem bestimmten Abbild übersetzt wird, aus der Definitionsmenge, dann ist dies ein \textbf{Urbild} von $y$.

Schließlich haben wir schon herausgefunden, dass verschiedene Elemente der Definitionsmenge verschiedene \textbf{Bilder} haben können. Eine Abbildung wird bei jeder Anwendung auf ein bestimmtes Element der Definitionsmenge angewandt. Dieses Element nennt man \textbf{Argument} der Abbildung.

\pagebreak
\begin{example}
    \parpic[r]{
        \begin{tikzpicture}[scale=.75]
            \fill[grayset] (-1.5,0) ellipse (0.7cm and 2cm);
            \fill[grayset] (1.5,0) ellipse (1.5cm and 2cm);
            %elements
            \node[blue!70] (x1) at (-1.5,0.7) {\LARGE $\bullet$};
            \node[orange] (x2) at (-1.5,-0.7) {$\bullet$};
            \node (y1) at (1.5,0.7) {1.50\,\euro};
            \node (y2) at (1.5,-0.7) {1.00\,\euro};
            %arrows
            \draw[blue!70,->] (x1) -- (y1);
            \draw[orange,->] (x2) -- (y2);
        \end{tikzpicture}
    }
    Das Preisschild an der Eisdiele hat wie jede Abbildung eine ganz bestimmte Zielmenge: Sie besteht aus den Preisen 1.00 \euro{} und 1.50 \euro, also ist die Zielmenge \[\textsc{MöglichePreise}\coloneqq\{\text{1.00\,\euro},\text{1.50\,\euro}\}.\]
    Die Abbildung, die vom Schild beschrieben wird, ist somit insgesamt eine Abbildung von der Definitionsmenge \textsc{Kugeln} zur Zielmenge \textsc{MöglichePreise}. Man sagt auch oft kürzer, sie ist eine Abbildung von \textsc{Kugeln} nach \textsc{MöglichePreise}. Das dementsprechende vollständige Diagramm zu dieser Abbildung ist rechts zu sehen.
    
    \picskip{0}Wertet man die Abbildung beispielsweise für das \textbf{Argument} \emph{große Kugel} aus, so erhält man das \textbf{Bild} 1.50\,\euro. Umgekehrt ist \emph{große Kugel} ein \textbf{Urbild} von 1.50\,\euro.
\end{example}
\end{document}