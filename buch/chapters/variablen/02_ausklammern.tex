\documentclass[../../main.tex]{subfiles}

\begin{document}
	In der Einleitung haben wir gesehen, dass wir dieselbe Rechnung durch verschiedene Terme darstellen können.
	Denselben Effekt können wir auch für das Rechnen mit Zahlen beobachten: Die Terme $2+3$ und $3+2$ haben
	wegen des \enquote{Vertauschungsgesetzes} (Kommutativität der Addition \mayberef) denselben Wert.

	Das Distributivgesetz, das du im ersten Kapitel dieses Buchs kennengelernt hast, sagt aus, dass $a\cdot b+a\cdot c=a\cdot (b+c)$ für beliebige Zahlen $a,b,c$ gilt.

	\begin{example}{}
		Mithilfe des Distributivgesetzes kannst du dir das Kopfrechnen vereinfachen. Zum Beispiel ist $12\cdot 8 + 12\cdot 2$ wesentlich
		einfacher im Kopf zu berechnen, wenn du das Distributivgesetz verwendest und
		\[12\cdot 8 + 12\cdot 2 = 12\cdot(8+2) = 12\cdot 10 = 120\]
		rechnest.
	\end{example}

	Du kannst das Distributivgesetz also verwenden, um einen Term, der zunächst einmal schwer zu berechnen oder anderweitig kompliziert ist, so umzuformen, dass du besser mit ihm arbeiten kannst.

	Wenn du es nun mit Termen zu tun hast, die Variablen enthalten, dann ändert sich überhaupt nichts daran. Denn das Distributivgesetz gilt für beliebige Zahlen, die du für $a,b$ und $c$ einsetzt. Jede Variable, die in einem Term wie
	\[6x+3x\]
	vorkommt, ist am Ende auch nur eine Zahl. Wir halten uns nur noch offen, \emph{welche} Zahl wir am Ende einsetzen. Doch alle Teile des letzten Terms sind einfach nur Zahlen. $x$ ist ebenso wie $6x$ und $3x$ eine Zahl. Deshalb hält uns nichts davon ab, auch hier das Distributivgesetz anzuwenden:
	\[6x+3x=(6+3)\cdot x=9x.\]

	Die Erkenntnis, dass $6x+3x=9x$ gilt, funktioniert nicht nur, wenn du mit Variablen rechnest. In ganz alltäglichen Beispielen wie dem folgenden machst du genau das gleiche, ohne darüber nachzudenken.

	\begin{example}{}
		\parpic[r]{
			\includegraphics[width=.2\textwidth]{images/apfel.png}
		}
		\picskip{3}
		Stell dir vor, du hast einen Apfelbaum und konntest von diesem Baum $6$ Äpfel ernten. Am nächsten Tag kaufst du 3 weitere Äpfel.
		Insgesamt hast du nun \[6\,\mtext{Äpfel}+3\,\mtext{Äpfel},\]
		was natürlich nichts anderes ist als $9\,\mtext{Äpfel}$. Überraschend ist diese Erkenntnis natürlich nicht, aber sie zeigt, dass wir auch dann normal mit Zahlen rechnen können, wenn jeweils die gleiche Variable dahinter steht.
	\end{example}

	Im Beispiel stand dort, wo vorher in unserem Term ein $x$ stand, nun das Wört \enquote{Äpfel}. Wenn du also $6x+3x$ zu $9x$ zusammenfasst, dann machst du schwierigeres als Äpfel zusammenzuzählen. Es ist also vollkommen egal, ob du Äpfel zusammenzählst oder Dinge, die Namen wie $x$, $n$ oder andere Variablen haben.

	\todo{\enquote{Einsetzen} von Werten für Variablen definieren}
	
	\begin{example}{}
		Wie in der Einführung hast du zwei Rechtecke, die nebeneinander gelegt
		ein neues Rechteck bilden und möchtest nun einen Ausdruck für die Gesamtfläche angeben.
		\begin{center}
            \begin{tikzpicture}
                \fill[orange!10, draw=orange, text=orange] (0,0) rectangle (2cm, 1cm) node[pos=.5]{$A_1$};
                \fill[blue!10, draw=blue, text=blue] (2cm,0cm) rectangle (3cm, 1cm) node[pos=.5]{$A_2$};
                \draw[decorate, decoration = {calligraphic brace, raise=5pt, amplitude=5pt}]
                    (0,0) -- (0,1)
                    node[pos=0.5,left=10pt,black]{$h$};
                \draw[decorate, decoration = {calligraphic brace, raise=5pt, amplitude=5pt}]
                    (0,1) -- (2,1)
                    node[pos=0.5,above=10pt,black]{$b_1$};
                \draw[decorate, decoration = {calligraphic brace, raise=5pt, amplitude=5pt}]
                    (2,1) -- (3,1)
                    node[pos=0.5,above=10pt,black]{$b_2$};
            \end{tikzpicture}
        \end{center}
		Dafür haben wir bereits die beiden Formeln $h\cdot b_1 + h\cdot b_2$ und $h\cdot (b_1+b_2)$ gefunden. Wir wissen bereits, dass diese beiden Terme gleich sein müssen, weil sie beide die gleiche Fläche beschreiben. Alternativ können wir auch schnell mit dem Distributivgesetz sehen, dass die Terme gleich sind:
		$h\cdot b_1+h\cdot b_2=h\cdot (b_1+b_2)$ ist genau die Aussage des Distributivgesetzes 
		\[a\cdot b+a\cdot c=a\cdot (b+c),\] 
		nur mit anderen Buchstaben.
		Wenn wir uns die beiden Terme einmal anschauen, dann fällt natürlich auf, dass wir im rechten Term eine Klammer haben, die links noch nicht stand. Diese Klammer ist entstanden, weil wir bemerkt haben, dass in beiden Summanden auf der linken Seite der Faktor $h$ vorkommt. Dadurch konnten wir das Distributivgesetz anwenden und $h$ vor die Klammer ziehen. Dazu sagt man auch, dass wir $h$ \textbf{ausgeklammert} haben.
		\if 0
		Welcher der beiden Terme für den Flächeninhalt ist jetzt aber besser?
		\begin{itemize}
			\item Wir setzen $b_1=2$ und $b_2=3$ in beide Formeln ein. Dann bekommen wir die Terme
				\[h\cdot 2 + h\cdot 3 \text{ bzw. } h\cdot(2+3).\]
				Während wir im linken Term keine Teilterme mehr zusammenfassen können, wissen wir im rechten Term, dass $2+3=5$.
				Somit können wir auch schreiben
				\[h\cdot 2 + h\cdot 3 \text{ bzw. } h\cdot 5.\]
				Hier ist aber ganz klar der rechte Term besser, weil wir $h$ nur noch an einer Stelle einsetzen müssen.
				Wir wissen ja, dass wir $h$ mit dem Distributivgesetz vor die Klammern ziehen können. Daher sagen wir
				im zweiten Term, wir haben \enquote{$h$ ausgeklammert}.
			\item Wir setzen $h=3$ und $b_1=2$ in beide Formeln ein. Dann bekommen wir die Terme:
				\[3\cdot2+3\cdot b_2 \text{ bzw. } 3\cdot(2+b_2).\]
				Hier sehen wir, dass der linke Term weiter vereinfacht werden kann:
				\[6+3\cdot b_2 \text{ bzw. } 3\cdot(2+b_2).\]
				Und in diesem Fall ist also der linke Term übersichtlicher.
		\end{itemize}
		\fi
	\end{example}

	Was machen wir also eigentlich, wenn wir das Distributivgesetz $a\cdot b+a\cdot b=a\cdot (b+c)$ verwenden? Im Grunde genommen ist es ganz einfach: 
	\begin{enumerate}
		\item Wir \textbf{suchen einen Faktor, der in jedem Summanden vorkommt}, hier also $a$ (denn er taucht sowohl im Summanden $a\cdot b$ als auch im Summanden $a\cdot c$ auf). 
		\item Wir schreiben den \textbf{gemeinsamen Faktor vor die Klammer} und das, was von den Summanden übrig bleibt, wenn man den Faktor $a$ weglässt, in die Klammer.
	\end{enumerate}
	\[{\color{orange}a}\cdot b + {\color{orange}a}\cdot c = {\color{orange}a\cdot (}b+c{\color{orange})}.\]
	Nun befindet sich der Faktor $a$ also außerhalb der Klammer. Man sagt deswegen, dass wir $a$ hier \textbf{ausgeklammert} haben. Wir können einen Faktor also dann ausklammern, wenn er in mehreren Summanden vorkommt.

	\begin{example}{}
		Im Term \[x^2+3x\] können wir den Faktor $x$ ausklammern, denn er kommt sowohl im Summanden $x^2$ als auch im Summanden $3x$ vor. In der Klammer bleibt dann $\frac{x^2}{x}+\frac{3x}{x}=x+3$ übrig. Wenn wir $x$ ausklammern, erhalten wir also den Term \[x\cdot (x+3).\]
	\end{example}
	
	Gemeinsame Faktoren können wir auch dann ausklammern, wenn wir mehr als zwei Summanden haben:
	\[{\color{orange}a}\cdot b + {\color{orange}a}\cdot c + {\color{orange}a}\cdot d = {\color{orange}a\cdot (}b+c{\color{orange})} + {\color{orange}a}\cdot d = {\color{orange}a\cdot (}b+c+d{\color{orange})}\]
	Dabei ist es natürlich auch möglich, Summanden zu ignorieren, die unseren Faktor nicht enthalten und den Faktor einfach nur aus den restlichen Summanden auszuklammern.
	\begin{example}{}
		Die ersten drei Summanden des Terms $4x+4y-4+3xy$ enthalten jeweils den Faktor $4$, der vierte allerdings nicht. Wir können den Faktor nun einfach nur aus den ersten drei Summanden ausklammern und den vierten Summand, also $3xy$, hinter das Ergebnis schreiben:
		\[\colorbrace{4x+4y-4}{4\text{ ausklammern}}+3xy=4\cdot (x+y-1)+3xy\]
	\end{example}

	Bisher haben wir immer nur von Addition und Multiplikation gesprochen. Wie sieht das aus, wenn wir subtrahieren oder dividieren?
	Weil eine Subtraktion von $a$ dasselbe ist wie die Addition von $(-a)$ und die Division durch $a$ dasselbe ist wie die Multiplikation mit $\frac{1}{a}$, können wir genauso ausklammern wie vorher.
	\begin{example}{}
		Wir starten mit dem Term $\left(\frac{3}{a}+\frac{b}{a}\right)$. In beiden Summanden teilen wir eine Zahl durch $a$. Wir können diese Division als Multiplikation mit $\frac{1}{a}$ betrachten und diesen Bruch entsprechend ausklammern:
		\[\left(\frac{3}{a}+\frac{b}{a}\right)=(3+b)\cdot \frac{1}{a}.\] 
		Dies können wir natürlich auch hübscher schreiben als $\frac{3+b}{a}$.
	\end{example}
	\todo{Schwere Aufgabe (?): Klammere aus: (3/{2a} + b/a) = (3/2+b)/a oder (3+2b)/{2a}}

	\todo{Schwere Aufgabe: Gebe eine Linearfaktorisierung für $(x^3-xy^2+x^2y-y^3) = (x+y)^2(x-y)$ an.}

	\begin{nutshell}{Ausklammern}
		Wenn in mehreren Summanden eines Terms ein gemeinsamer Faktor vorkommt, dann können wir ihn mithilfe des Distributivgesetzes aus den Termen herausziehen und vor eine Klammer schreiben:
		\[{\color{orange}a}\cdot b + {\color{orange}a}\cdot c = {\color{orange}a\cdot (}b+c{\color{orange})}.\]
		Diese Umformung nennen wir \textbf{Ausklammern}. Das Distributivgesetz funktioniert in Termen mit Variablen also genauso wie bei Termen, in denen nur Zahlen vorkommen.
	\end{nutshell}
\end{document}