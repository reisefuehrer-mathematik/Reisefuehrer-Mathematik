\documentclass[../../main.tex]{subfiles}

\begin{document}

\subsection*{Funktionen mengentheoretisch definieren}


\subsection*{Injektionen, Surjektionen und Bijektionen}
\pagecolor{violet!20}
\label{advanced:bijektion}
Man defniert für Funktionen bestimmte Eigenschaften, die sie zu injektiven, surjektiven bzw. bijektiven Funktionen machen können.

\begin{definition}{}
    Eine Funktion $f\colon U\rightarrow V$ heißt \textbf{surjektiv}, falls für jedes $v\in V$ ein $u\in U$ mit $f(u)=v$ existiert, d.h. wenn jedes Element der Bildmenge mindestens ein Urbild hat.
    
    Sie heißt \textbf{injektiv}, falls aus $f(u)=f(u')$ für beliebige $u,u'\in U$ stets folgt, dass $u=u'$ gilt. 
    
    Eine Funktion heißt bijektiv, falls sie \textbf{surjektiv} und \textbf{injektiv} ist.
\end{definition}

Graphisch sieht man die Bedeutung der Begriffe \textbf{surjektiv}, \textbf{injektiv} und \textbf{bijektiv} am besten am Zuordnungsdiagramm:
    
\begin{multicols}{3}\centering
    \begin{tikzpicture}[scale=.75]
        \draw[grayset] (-1.5,0) ellipse (0.7cm and 2cm);
        \draw[grayset] (1.5,0) ellipse (0.7cm and 2cm);
    
        \node (x1) at (-1.5,1.4) {$\bullet$};
        \node (x2) at (-1.5,0.5) {$\bullet$};
        \node (x3) at (-1.5,-0.4) {$\bullet$};
        \node (x4) at (-1.5,-1.3) {$\bullet$};
        \node (y1) at (1.5,0.7) {$\bullet$};
        \node (y2) at (1.5,-0.2) {$\bullet$};
        \node (y3) at (1.5,-1.2) {$\bullet$};
    
        \draw[->] (x1) -- (y2);
        \draw[->] (x2) to[bend right] (y1);
        \draw[->] (x3) to[bend right] (y2);
        \draw[->] (x4) -- (y3);
    \end{tikzpicture}
    
    \begin{tikzpicture}[scale=.75]
        \draw[grayset] (-1.5,0) ellipse (0.7cm and 2cm);
        \draw[grayset] (1.5,0) ellipse (0.7cm and 2cm);
    
        \node (x1) at (-1.5,0.75) {$\bullet$};
        \node (x2) at (-1.5,-0.75) {$\bullet$};
        \node (y1) at (1.5,0.7) {$\bullet$};
        \node (y2) at (1.5,-0.2) {$\bullet$};
        \node (y3) at (1.5,-1.2) {$\bullet$};
    
        \draw[->] (x1) -- (y2);
        \draw[->] (x2) to[bend right] (y1);
    \end{tikzpicture}
    
    \begin{tikzpicture}[scale=.75]
    \draw[grayset] (-1.5,0) ellipse (0.7cm and 2cm);
    \draw[grayset] (1.5,0) ellipse (0.7cm and 2cm);

    \node (x1) at (-1.5,0.7) {$\bullet$};
    \node (x2) at (-1.5,-0.2) {$\bullet$};
    \node (x3) at (-1.5,-1.1) {$\bullet$};
    \node (y1) at (1.5,0.7) {$\bullet$};
    \node (y2) at (1.5,-0.2) {$\bullet$};
    \node (y3) at (1.5,-1.2) {$\bullet$};

    \draw[->] (x1) -- (y3);
    \draw[->] (x2) to[bend right] (y1);
    \draw[->] (x3) to[bend right] (y2);

    \draw[maincolor, <-] (-1.1,1) to[bend right] (-0.2,1.7);
    \draw[maincolor, <-] (1.2,-1.5) to[bend right] (0.3,-1.85);

    \node[maincolor, text width=15mm, align=center] at (0,-0.8) {\tiny Funktion};

    \node[maincolor, text width=15mm, align=center] at (-0.2,1.77) {\scriptsize menge};
    \node[maincolor, text width=15mm, align=center] at (-0.2,1.85) {\scriptsize Definitions-};

    \node[maincolor, text width=15mm, align=center] at (0.3,-1.98) {\scriptsize Zielmenge};;
\end{tikzpicture}
\end{multicols}

Wie viele Urbilder ein Element der Bildmenge hat, sieht man an der Anzahl der Pfeile, die auf dieses Element zeigen.

Im linken Diagramm hat jedes Element der Bildmenge mindestens ein Urbild, die Funktion ist also surjektiv. Da das 
mittlere Element allerdings zwei Urbilder hat, lässt sich nicht sagen, dass jedes Element \emph{genau} ein Urbild hat. 
Die Funktion ist daher weder injektiv noch bijektiv.

Im mittleren Beispiel hat das untere Element der Bildmenge kein Urbild. Daher ist die Funktion zwar injektiv (kein Element hat mehrere Urbilder), aber nicht surjektiv und deswegen auch nicht bijektiv.

Im rechten Bild hat jedes Element genau ein Urbild. Daher ist die Funktion bijektiv und damit erst recht surjektiv und injektiv.

\begin{example}{}
    Die Funktion \textsc{MitGelbMischen}, die in einigen Beispielen im Kapitel vorgekommen ist, ist bijektiv. Das liegt daran, dass es für jede Mischfarbe (grün, orange und gelb) nur genau eine Grundfarbe gibt, die man mit gelb mischen, um die jeweilige Mischfarbe zu erhalten.
\end{example}

Wenn zwischen zwei Mengen eine Funktion $f\colon U\rightarrow V$ existiert, dann lässt sich die Kardinalität von $U$ und $V$ vergleichen -- abhängig davon, ob $f$ eine Injektion, Surjektion oder Bijektion ist. In diesen Fällen lässt sich eine Aussage darüber treffen, welche Menge mehr Elemente enthält.

\begin{theorem}{}
    Es sei $f\colon U\rightarrow V$ eine Funktion. Dann gilt:
    \begin{enumerate}
        \item Falls $f$ injektiv ist, dann gilt $|U|\leq |V|$.
        \item Falls $f$ surjektiv ist, dann gilt $|U|\geq |V|$.
        \item Falls $f$ bijektiv ist, dann gilt $|U|=|V|$.
    \end{enumerate}
\end{theorem}

Das bedeutet insbesondere, dass das Definieren einer Bijektion zwischen zwei Mengen $U$ und $V$ eine Methode ist, um zu 
zeigen, dass die beiden Mengen gleich viele Elemente besitzen. Damit können bijektive Funktionen zu einer Methode 
werden, die Elemente einer Menge zu zählen: Man gibt zwischen einer Menge $U$, deren Elemente man zählen möchte, und 
einer Menge $V$, von der man weiß, wie viele Elemente sie enthält, eine Bijektion an. Dadurch ist sofort klar, wie 
viele Elemente $V$ besitzt (nämlich genauso viele wie $U$). Dies ist insbesondere auch eine Möglichkeit, die Kardinalität 
unendlicher Mengen zu vergleichen, wie wir uns als nächstes anschauen werden.

\subsection*{Abzählbarkeit und Überabzählbarkeit}
In diesem Abschnitt möchten wir uns anschauen, wie wir die Kardinalität unendlicher Mengen miteinander vergleichen können.
Dies klingt erstmal langweilig, da alle unendlichen Mengen gleich groß scheinen -- nämlich unendlich. Tatsächlich ist
das aber nicht ganz richtig -- unendliche Mengen können also \enquote{verschieden unendlich} sein.

\subsection*{Isomorphie}
Sowohl in der Mathematik als auch an vielen anderen Stellen kommen häufig verschiedene Objekte vor, die zwar erst einmal 
unterschiedlich sind, aber wenn man genauer hinsieht viele gemeinsame Eigenschaften haben. Zum Beispiel sind die 
Hauptgottheiten in der römischen und griechischen Mythologie recht ähnlich zu einander und entsprechen sich 
größtenteils. Etwa ist Zeus der Göttervater bei den Griechen, während Jupiter dies bei den Römern ist. Obwohl Jupiter 
und Zeus erst einmal verschiedene Gottheiten sind, könnte man also sagen, dass sie einander entsprechen, weil sie die 
gleiche Rolle in der Religion einnehmen.

Auch in der Mathematik lassen sich solche Fälle antreffen. Im weiterführenden Wissen auf Seite \ref{graphs} haben wir 
erklärt, wie Graphen (\emph{nicht} gemeint sind hier Funktionsgraphen!) definiert sind. Sie bestehen aus verschiedenen 
Knoten, die durch Kanten miteinander verbunden sein können. Beispielsweise stellen die folgenden Abbildungen Graphen dar.
\begin{center}
\begin{multicols}{3}
    \tikz{
        \node (v1) at (0:2) {$v_1$};
        \node (v2) at (72:2) {$v_2$};
        \node (v3) at (144:2) {$v_3$};
        \node (v4) at (216:2) {$v_4$};
        \node (v5) at (288:2) {$v_5$};
        \draw[-latex] (v1) -- (v2);
        \draw[-latex] (v3) -- (v2);
        \draw[-latex] (v1) -- (v5);
        \draw[-latex] (v5) -- (v4);
        \draw[-latex] (v4) -- (v3);
        \node at (-90:2.5) {$G_1$};
    }
    \tikz{
        \node (v1) at (0:2) {$a$};
        \node (v2) at (72:2) {$b$};
        \node (v3) at (144:2) {$c$};
        \node (v4) at (216:2) {$d$};
        \node (v5) at (288:2) {$e$};

        \draw[-latex] (v1) -- (v3);
        \draw[-latex] (v5) -- (v3);
        \draw[-latex] (v5) -- (v2);
        \draw[-latex] (v2) -- (v4);
        \draw[-latex] (v4) -- (v1);
        \node at (-90:2.5) {$G_2$};
    }
    \tikz{
        \node (v1) at (-2,2) {$g$};
        \node (v2) at (0,2) {$a$};
        \node (v3) at (2,2) {$h$};
        \node (v4) at (-1,0) {$r$};
        \node (v5) at (1,0) {$p$};
        \draw[-latex] (v1) -- (v4); 
        \draw[-latex] (v4) -- (v2);
        \draw[-latex] (v5) -- (v2);
        \draw[-latex] (v5) -- (v3);
        \draw[-latex] (v3) to[in=90,out=90] (v1);
        \node at (-90:0.9)  {$G_3$};
    }
\end{multicols}
\end{center}
Offensichtlich sehen all diese Graphen verschieden aus und da sie verschiedene Knoten haben (der Knoten $v_1$ aus dem 
linken Graph kommt zum Beispiel in keinem anderen Graphen vor), sind sie auch wirklich verschieden. Dennoch lassen sich 
gewisse Gemeinsamkeiten feststellen. Zum Beispiel haben alle Graphen einen Knoten, in dem zwei Pfeilspitzen enden: $v_2$
im ersten Graph, $c$ im zweiten Graph und $a$ im dritten Graph. Wenn wir auf diese Weise weiter Knoten suchen, die sich 
entsprechen, dann fällt uns auf, dass $G_2$ und $G_3$ sich genauso wie der Graph $G_1$ zeichnen lassen, wenn
wir die Punkte etwas verschieben. Schauen wir uns das einmal für den Graphen $G_2$ an (die nächste Abbildung zeigt 
das Vorgehen Schritt für Schritt).

\begin{center}
    \begin{multicols}{3}
    \tikz{
        
        \node (v1) at (180:2) {$a$};
        \node (v2) at (72:2) {$b$};
        \node (v3) at (144:2) {$c$};
        \node (v4) at (216:2) {$d$};
        \node (v5) at (288:2) {$e$};

        \draw[-latex] (v1) -- (v3);
        \draw[-latex] (v5) -- (v3);
        \draw[-latex] (v5) -- (v2);
        \draw[-latex] (v2) -- (v4);
        \draw[-latex] (v4) -- (v1);
    }
    \tikz{
        
        \node (v1) at (180:2) {$a$};
        \node (v2) at (252:2) {$b$};
        \node (v3) at (144:2) {$c$};
        \node (v4) at (216:2) {$d$};
        \node (v5) at (288:2) {$e$};

        \draw[-latex] (v1) -- (v3);
        \draw[-latex] (v5) -- (v3);
        \draw[-latex] (v5) -- (v2);
        \draw[-latex] (v2) -- (v4);
        \draw[-latex] (v4) -- (v1);
    }
    \tikz{
        
        \node (v1) at (144:2) {$a$};
        \node (v2) at (288:2) {$b$};
        \node (v3) at (72:2) {$c$};
        \node (v4) at (216:2) {$d$};
        \node (v5) at (0:2) {$e$};

        \draw[-latex] (v1) -- (v3);
        \draw[-latex] (v5) -- (v3);
        \draw[-latex] (v5) -- (v2);
        \draw[-latex] (v2) -- (v4);
        \draw[-latex] (v4) -- (v1);
    }
    \end{multicols}
\end{center}
Wir haben hier zunächst den Knoten $a$ nach links geschoben, dann $b$ nach unten geschoben und schließlich alle Knoten 
ein wenig zurechtgerückt. Nichts davon hat den Graphen wirklich verändert -- wir haben nur die Punkte verschoben 
(den Graphen also anders gezeichnet), aber weder die Knoten noch die Kantenrelation wurde verändert. 

Unser Ergebnis ist ein Graph, der genauso aussieht wie der Graph ganz links. 
Die Graphen sind also eigentlich identisch -- wir haben nur die Knoten anders genannt. Der Punkt $e$ entspricht nun 
$v_1$, $c$ entspricht $v_2$ und so weiter. Diese Entsprechung können wir durch eine bijektive Funktion
\[\phi\colon\{v_1,v_2,v_3,v_4,v_5\}\rightarrow\{a,b,c,d,e\}\]
mit
\[\phi(v_1)=e,\phi(v_2)=c,\phi(v_3)=a,\phi(v_4)=d,\phi(v_5)=b\]
ausdrücken. Diese Funktion identifiziert die Knoten aus $G_1$ also derart mit denen aus $G_2$, dass jede Kante $(u,v)$ 
zwischen zwei Knoten $u,v$ aus $G_1$ auch zwischen den Bildern $\phi(u)$ und $\phi(v)$ entspricht, d.h.
\[(u,v)\in E_1\text{ genau dann, wenn }(\phi(u),\phi(v))\in E_2.\]
Die Funktion $\phi$ ist eine Art Beweis dafür, dass die beiden Graphen eigentlich identisch sind, da sie die gleiche 
Struktur haben (die Struktur sind hier die Stellen, an denen Kanten existieren). Graphen, die diese Eigenschaften haben,
nennt man \textbf{isomorph} und die Funktion $\phi$ entsprechend einen \textbf{Isomorphismus}. Zwei isomorphe Graphen
sind also eigentlich gleich, aber ggf. heißen ihre Punkte anders. Sie haben die gleiche Struktur.

\begin{definition}{Isomorphie von Graphen}
    Zwei Graphen $G_1=(V_1,E_1)$ und $G_2=(V_2,E_2)$ heißen \textbf{isomorph}, falls eine bijektive Funktion 
    $\phi\colon V_1\rightarrow V_2$ existiert, sodass
    \[(u,v)\in E_1\text{ genau dann, wenn }(\phi(u),\phi(v))\in E_2\]
    für alle $u,v\in V_1$ gilt. In diesem Fall schreiben wir auch $G_1\isom G_2$ und nennen $\phi$ einen \textbf{Isomorphismus} zwischen $G_1$
    und $G_2$.
\end{definition}

Isomorphie ist ein allgemeines Konzept, das nicht nur für Graphen funktioniert. Wenn wir schauen möchten, ob zwei 
Strukturen isomorph sind, müssen wir also immer eine Bijektion zwischen ihnen finden,
die die wichtigen Strukturen (bei uns also die Kantenrelation) beibehält. Das bedeutet, dass wir uns, wenn wir Isomorphie für
andere Strukturen definieren möchten, zunächst überlegen müssen, welche Eigenschaften so wichtig sind, dass ein Isomorphismus sie beibehalten
sollte. Du wirst in den Aufgaben zu diesem Abschnitt einige weitere Beispiele für Isomorphiebegriffe in der Mathematik finden.

\subsection*{Existenz der Umkehrfunktion}
Ob eine Funktion bijektiv ist, steht im direkten Zusammenhang damit, ob zu ihr eine Umkehrfunktion existiert. Es lässt sich nämlich beweisen:

\begin{theorem}{Existenz der Umkehrfunktion}
    Zu einer Funktion $f\colon U\rightarrow V$ existiert genau dann eine Umkehrfunktion, wenn $f$ bijektiv ist.
\end{theorem}

Der Beweis zu diesem Satz befindet sich am Ende dieses Kapitels auf Seite \pageref{proof:existenceOfInverseMap}.
Es ist meistens aufwendig, auf direktem Wege zu prüfen, ob eine Funktion $f$ bijektiv ist. Wir können aber auch einfach
eine Umkehrfunktion $f^{-1}$ angeben. Daraus folgt dann, dass $f$ bijektiv sein muss.

\begin{example}{}
\end{example}

\begin{proof}
    \label{proof:existenceOfInverseMap}
    Es soll bewiesen werden, dass zu einer Funktion eine Umkehrfunktion existiert genau dann, wenn die Funktion bijektiv ist. Um diese Äquivalenz zu zeigen, zeigen wir, dass aus der Existenz der Umkehrfunktion von $f$ folgt, dass $f$ bijektiv ist (d.h. aus der linken Aussage folgt die rechte). Außerdem zeigen wir, dass daraus, dass $f$ bijektiv ist, folgt, dass es eine Umkehrfunktion zu $f$ gibt (d.h. aus der rechten Aussage folgt die linke). Insgesamt folgt dann, dass beide Aussagen äquivalent sind.
    
    \enquote{$\Leftarrow$}:
    Falls $f$ eine bijektive Funktion ist, dann gibt es zu jedem $v\in V$ ein eindeutig bestimmtes Element $u_v\in U$ mit $f(u)=v$ (wenn ein solches Element nicht existieren würde, dann wäre $f$ nicht surjektiv und wenn es nicht eindeutig wäre, dann wäre $f$ nicht injektiv). 
    
    Dann ist die Funktion $g\colon V\rightarrow U$, die definiert ist durch $g(v)=u_v$ die gewünschte Umkehrfunktion von $f$. Gilt nämlich $f(u')=v'$ für ein $u'\in U,v'\in V$, dann ist $g(f(u'))=g(v')=u'_v$. Weil $u'_{v'}$ eindeutig ist und die Eigenschaft $f(u'_{v'})=v'$ hat, muss $u'_{v'}=u'$ gelten. Damit ist $g(f(u'))=u'$ und $g\circ f$ die Identität auf $U$. Auf die gleiche Weise kann gezeigt werden, dass $f\circ g$ die Identität auf $V$ ist.
    
    \enquote{$\Rightarrow$}:
    Sei $g$ eine solche Umkehrfunktion von $f$. Dann existiert zu jedem $v\in V$ ein eindeutiges $u\in U$ mit $g(v)=u$. Angenommen, $f$ sei nicht injektiv. Dann existieren $u_1,u_2\in U$ mit $u_1\neq u_2$, sodass $f(u_1)=f(u_2)=v$ für ein $v\in V$ gilt. Dann ist $g(f(u_1))=g(v)=g(f(u_2))$. Nun kann $g$ aber keine Umkehrfunktion von $f$ mehr sein, denn es gilt nun entweder $g(f(u_1))\neq u_1$ oder $g(f(u_2))\neq u_2$. Also muss $f$ injektiv sein.
    
    Falls $f$ nicht surjektiv ist, dann existiert ein $v\in V$, sodass $v$ kein Urbild bzgl. $f$ hat. Entsprechend ist $f(g(v))\neq v$. Damit ist aber $f\circ g\neq id_V$ und $g$ keine Umkehrfunktion von $f$. Also muss $f$ surjektiv sein.
\end{proof}

\newpage
\pagecolor{white}

\end{document}