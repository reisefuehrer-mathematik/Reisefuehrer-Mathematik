\documentclass[../../main.tex]{subfiles}

\begin{document}

Variablen bisher implizit Zahlen. Aber: Besser, man könnte einen Wertevorrat angeben

\begin{example}{}
    Beispiel für Wertevorräte bzw Sätze mit solchen (z.B. fehlt noch ein Besteckstück-> was könnte das sein?)
\end{example}

Was macht eine Menge aus? (ihre Elemente)

\begin{example}{}
    Grundfarben inkl. Darstellung als Sack
\end{example}

Wie kann man den Spaß aufschreiben? Erstmal nur explizit

\begin{definition}{}
    Für die Definition einer Menge braucht man ne Menge Mathematik ()
\end{definition}

TODO: Wo kommt die implizite Schreibweise hin? Wäre vermutlich etwas viel das auch noch in die Einführung zu packen

\end{document}